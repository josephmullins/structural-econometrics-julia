\documentclass[xcolor=dvipsnames,handout,c]{beamer}
%\documentclass[xcolor=dvipsnames]{beamer}
\usepackage{amssymb}
\usepackage{amsmath}
%\usepackage{mathpazo}
\usepackage{multimedia}
\usepackage{graphicx}
\usepackage{rotating}
\usepackage{pgfpages}

\usetheme{default}
\usepackage{graphicx}
\usepackage{amsthm}
\usepackage{bm}
\newcommand\ov{\overline}
\newcommand\un{\underline}
\newcommand\BB{\mathbb}
\newcommand\EE{\mathbb{E}}
\newcommand\mc{\mathcal}
\newcommand\ti{\tilde}
\newcommand\h{\hat}
\newcommand\beq{\begin{equation}}
\newcommand\eeq{\end{equation}}
\newcommand\barr{\begin{array}}
\newcommand\earr{\end{array}}
\newcommand\bfp{\mathbf{p}}
\newcommand\pder[2]{\frac{\partial #1}{\partial #2}}
%\DeclareMathOperator*{\plim}{plim}
\setbeamertemplate{itemize items}[circle]
\setbeamertemplate{navigation symbols}{}

\newcommand\eps{\epsilon}
\newcommand\veps{\varepsilon}


\setcounter{MaxMatrixCols}{10}
\mode<presentation>
\newtheorem{acknowledgement}[theorem]{Acknowledgement}
\newtheorem{algorithm}[theorem]{Algorithm}
\newtheorem{assumption}[theorem]{Assumption}
\newtheorem{axiom}[theorem]{Axiom}
\newtheorem{case}[theorem]{Case}
\newtheorem{claim}[theorem]{Claim}
\newtheorem{conclusion}[theorem]{Conclusion}
\newtheorem{condition}[theorem]{Condition}
\newtheorem{conjecture}[theorem]{Conjecture}
\newtheorem{criterion}[theorem]{Criterion}
\newtheorem{exercise}[theorem]{Exercise}
\newtheorem{notation}[theorem]{Notation}
\newtheorem{proposition}[theorem]{Proposition}
\newtheorem{remark}[theorem]{Remark}
\newtheorem{summary}[theorem]{Summary}
%\newenvironment{stepenumerate}{\begin{enumerate}[<+->]}{\end{enumerate}}
%\newenvironment{stepitemize}{\begin{itemize}[<+->]}{\end{itemize} }
%\newenvironment{enumerate}{\begin{enumerate}[<+-| alert@+>]}{\end{enumerate}}
%\newenvironment{stepitemizewithalert}{\begin{itemize}[<+-| alert@+>]}{\end{itemize} }
\beamertemplatetransparentcoveredhigh
\usefonttheme[onlymath]{serif}
%\usetheme[height=7mm]{Boadilla}
\setbeamercovered{dynamic}
\usecolortheme[named=Blue]{structure}
\def\hilite<#1>{\temporal<#1>{\color{black!50}}{\color{blue}}{\color{black}}}
\setbeamertemplate{navigation symbols}{}
\setbeamertemplate{footline}[frame number]

\begin{document}

\title[Applied Econometrics]{Applied Econometrics: Extremum Estimators}
\author[Mullins]{Joseph Mullins \\
\bigskip \bigskip Econ 8208}
\institute{University of Minnesota}
\date{Spring 2020}
\maketitle

\begin{frame}\frametitle{Introduction}
  \begin{itemize}
  \item In the Fall you extensively studied GMM.
  \item Now look at how GMM falls within broad class of estimators.
  \item Goal: give you the tools to choose the right estimation method for your research.
  \item Understand tradeoffs for different methods.
  \item Apply these to different settings:
    \begin{itemize}
    \item[-] Discrete choice.
    \item[-] Search models.
    \item[-] Dynamic models and panel data.
    \item[-] ``Causal inference''.
    \end{itemize}
  \end{itemize}
\end{frame}


\begin{frame}

\frametitle{Example: Maximum Likelihood Estimator}

\begin{itemize}

\item Example: $y_{i}\in \{0,1\}$: employment status.

\item We are interested in understanding the importance of factors $x_{i}$
(e.g., wage offers, skills, family characteristics) affecting the
probability $\Pr (y_{i}=1|x_{i})$, which is nonlinear function of $x_{i}.$

\item One popular approach: maximum likelihood.

\begin{itemize}
\item maximum likelihood is an approach to estimate the model fully
exploiting individual level data.
\end{itemize}
\end{itemize}

\end{frame}%

\begin{frame}%

\frametitle{Maximum Likelihood Estimator}

\begin{itemize}
\item Specifically, let $f(y_{i}|\mathbf{x}_{i},\theta )$ be the likelihood
of $y_{i}$ conditional on $\mathbf{x}_{i}$ under parameter $\theta .$

\begin{itemize}
\item (example) suppose that $\Pr (y_{i}=1|x_{i})=\Phi (\mathbf{x}_{i}\beta
) $ and $\Pr (y_{i}=0|x_{i})=1-\Phi (\mathbf{x}_{i}\beta ).$ Then, the
likelihood is
\begin{equation*}
f(y_{i}|\mathbf{x}_{i},\theta )=\Phi (\mathbf{x}_{i}\beta )^{y_{i}}\left(
1-\Phi (\mathbf{x}_{i}\beta )\right) ^{1-y_{i}}.
\end{equation*}
\end{itemize}

\item Maximum likelihood estimator is the solution to
\begin{equation*}
\hat{\theta}_{ML}=\arg\max_{\theta }\prod_{i=1}^{n}f(y_{i}|\mathbf{x}_{i},\theta ).
\end{equation*}

\item This is equivalent to maximize the log likelihood:%
\begin{equation*}
\hat{\theta}_{ML}=\arg\max_{\theta }\sum_{i=1}^{n}\log (f(y_{i}|\mathbf{x}_{i},\theta )).
\end{equation*}

\item \textbf{Crucial}: what are the asymptotic properties of $\hat{\theta}_{ML}$?
\end{itemize}

\end{frame}%

\begin{frame}%

\frametitle{Our Plan}

\begin{itemize}
\item We first define two general classes of estimator called (1) {\color{blue}extremum
estimator} and (2) {\color{blue}M-estimator} (an important subclass).

\item Then, study asymptotic properties:

\begin{itemize}
\item[-] consistency

\item[-] asymptotic normality
\end{itemize}

\item Then study asymptotic properties of {\color{blue}ML}.
\item Then we introduce other techniques, often well-suited to complicated models:
  \begin{itemize}
  \item[-] {\color{blue}Minimum Distance}
  \item[-] {\color{blue} Simulated Method of Moments/Indirect Inference/Simulated ML}
  \end{itemize}
\item We discuss the {\color{blue}Bootstrap}, a convenient method for inference.
\end{itemize}

\end{frame}%

\begin{frame}%

\frametitle{Extremum Estimator}

\begin{itemize}
\item Extremum estimators comprises a wide class of estimation strategies.
They are defined as follows:

\begin{definition}
Extremum Estimators

$\hat{\theta}$ is an extremum estimator iff
\begin{equation*}
\hat{\theta}=\arg \max_{\theta }Q_{n}(\theta )\text{ }\theta \in \Theta
\subset R^{p}.
\end{equation*}%
for some $Q_{n}(\cdot ).$
\end{definition}
\end{itemize}

\end{frame}%

% \begin{frame}%

% \frametitle{Extremum Estimator: Existence}

% \begin{itemize}
% \item The following are sufficient conditions for the existence of extremum
% estimators:

% \begin{enumerate}
% \item $\Theta $ is compact;

% \item $Q_{n}(\cdot )$ is continuous in $\theta ;$

% \item $Q_{n}(\theta )$ is a measurable function of the data for any fixed $%
% \theta .$
% \end{enumerate}
% \end{itemize}
% \end{frame}%
\begin{frame}%

\frametitle{Extremum Estimator: Examples}

\begin{itemize}
\item (M-estimator)%
\begin{equation*}
Q_{n}(\theta )=\frac{1}{n}\sum_{i=1}^{n}m(\mathbf{w}_{i},\theta )
\end{equation*}

\item (GMM estimator)%
\begin{equation*}
Q_{n}(\theta )=-\frac{1}{2}\mathbf{g}_{n}^{\prime }(\theta )\mathbf{\hat{W}g}%
_{n}(\theta )
\end{equation*}%
where%
\begin{equation*}
\mathbf{g}_{n}(\theta )=\frac{1}{n}\sum g(\mathbf{w}_{i},\theta )
\end{equation*}
\item (Minimum Distance Estimator)
  \[
    Q_{n}(\theta,\hat{\pi}_n)=-\frac{1}{2}\psi^{\prime}(\theta,\hat{\pi}_n)\mathbf{\hat{W}}\psi(\theta,\hat{\pi}_n)
\]
where
\[ \sqrt{n}\hat{\pi}_n\rightarrow_d\mathcal{N}(\pi_0,\Sigma)\qquad \psi(\theta_0,\pi_0)=0
\]
\end{itemize}

\end{frame}%

\begin{frame}%

\frametitle{Examples of M-estimator}

\begin{itemize}
\item Maximum Likelihood:
\begin{equation*}
m(\mathbf{w}_{i},\theta )=\log (f(y_{i}|\mathbf{x}_{i},\theta ))
\end{equation*}

\item Nonlinear least squares%
\begin{equation*}
m(\mathbf{w}_{i},\theta )=\left( y_{i}-\varphi (\mathbf{x}_{i},\theta
)\right) ^{2}.
\end{equation*}
\item Exercise: M-estimator is a special case of GMM.
\end{itemize}
\end{frame}%

\begin{frame}%

\frametitle{Consistency}

\begin{itemize}
\item Suppose that

\begin{enumerate}
\item $Q_{n}(\theta )\rightarrow _{p}Q_{0}(\theta ),$ $\forall \theta .$

\item $\theta _{0}\,$\ is the maximizer of $Q_{0}(\theta )\,.$
\end{enumerate}

\item One could be tempted to infer that $\hat{\theta}\rightarrow _{p}\theta
_{0}...$

\item But these are not sufficient conditions.
\end{itemize}

\end{frame}%

\begin{frame}%

\frametitle{Identification}

\begin{itemize}
\item If the solution of $\max Q_{0}(\theta )$ is not unique, then it may
not converge to $\theta _{0}!$

\item We need to impose that $\theta _{0}\,$\ is the unique maximizer of $%
Q_{0}(\theta ).$
\item Alternative: \emph{partial identification}.
\end{itemize}

\end{frame}%

\begin{frame}%

\frametitle{Uniform Convergence}

\begin{itemize}
\item It maye be that $Q_{n}(\theta )\rightarrow
_{p}Q_{0}(\theta )$ but NOT $\hat{\theta}\rightarrow _{p}\theta _{0}.$

\item To guarantee this, we need to impose the stronger notion of
convergence:%
\begin{equation*}
\sup_{\theta \in \Theta }\left\vert Q_{n}(\theta )-Q_{0}(\theta )\right\vert
\rightarrow _{p}0.
\end{equation*}
\item Exercise: think of an example to show necessity ($Q_n(\theta) = \mathbf{1}\{\theta=\frac{n}{1+n}\} + 1-\theta^2$)
\end{itemize}

\end{frame}%

\begin{frame}%
\frametitle{Consistency with compactness}

\begin{theorem}
Suppose that

(1) $\Theta $ is a compact subset $R^{p}.$

(2) $Q_{n}(\theta )$ is continuous in $\theta ,\forall (\mathbf{w}%
_{i})_{i=1}^{n}$

(3) $Q_{n}(\theta )$ is a measurable function of the data, $\forall \theta
\in \Theta .$

If there is $Q_{0}(\cdot )$ such that

(a) (identification) $Q_{0}(\cdot )$ is uniquely maximized on $\Theta $ at $%
\theta _{0}\in \Theta .$

(b) (uniform convergence) $Q_{n}(\cdot )$ converges uniformly in probability
to $Q_{0}(\cdot )$ (in other words, assume that $\sup_{\theta \in \Theta
}\left\vert Q_{n}(\theta )-Q_{0}(\theta )\right\vert \rightarrow _{p}0.)$

then%
\begin{equation*}
\hat{\theta}\rightarrow _{p}\theta _{0}.
\end{equation*}
\end{theorem}

\end{frame}%

\begin{frame}\frametitle{Proof Sketch}
\begin{enumerate}
\item Show that $Q_0(\hat{\theta}) > Q_0(\theta_0) - \epsilon$ for any $\epsilon$ with prob approaching 1 (need uniform convergence)
\item For any small neighborhood around $\theta_0$, $\mathcal{N}$, compactness + continuity gives $\sup_{\theta\in\Theta\cap\mathcal{N}^c}Q_0(\theta)<Q_0(\theta_0)$
\item Choose $\epsilon = Q_0(\theta_0)-\sup_{\theta\in\Theta\cap\mathcal{N}^c}Q_0(\theta)$
\item This implies $\hat{\theta}\in\mathcal{N}$.
\end{enumerate}
\end{frame}

\begin{frame}

\frametitle{Consistency (w/o compactness)}

\begin{theorem}
Suppose that

(1) $\theta _{0}\in int(\Theta )$

(2)$\ Q_{n}(\theta )$ is concave in $\theta ,\forall (\mathbf{w}%
_{i})_{i=1}^{n}$

(3) $Q_{n}(\theta )$ is a measurable function of the data, $\forall \theta
\in \Theta .$

If there is $Q_{0}(\cdot )$ such that

(a) (identification) $Q_{0}(\cdot )$ is uniquely maximized on $\Theta $ at $%
\theta _{0}\in \Theta .$

(b) (pointwise convergence) $Q_{n}(\cdot )\rightarrow _{p}Q_{0}(\cdot
),\forall \theta $

then%
\begin{equation*}
\hat{\theta}\rightarrow _{p}\theta _{0}.
\end{equation*}
\end{theorem}

%Concave+pointwise convergence $\rightarrow $ uniform convergence <- this is false

\end{frame}%
\begin{frame}\frametitle{Proof Sketch}
    \begin{enumerate}
    \item Theorem: concavity implies uniform convergence on any compact subset of $\Theta$ (e.g. Rockafellar (1970))
    \item By $\theta_0\in int(\Theta)$ + continuity + concavity, $\exists$ $\epsilon$ s.t. $C = \{\theta: Q_0(\theta)\geq Q_0(\theta_0)-\epsilon\}$ is closed and bounded.
    \item $\tilde{\theta}_n=\arg\max_{\theta\in C} Q_n(\theta)$ is consistent by previous theorem.
    \item For any $\theta\notin C$, $Q_n(\theta_0) > Q_n(\theta)$ w.p.a. 1 by pointwise convergence.
    \item So $\tilde{\theta}_n=\hat{\theta}_n$ w.p.a. 1.
    \end{enumerate}
  \end{frame}

\begin{frame}
\frametitle{More on Uniform Convergence}
For $M$-estimators, $Q_n(\theta) = \frac{1}{n}\sum m(\mathbf{w}_i,\theta)$:
\begin{itemize}
\item Note that if we can use the Ergodic theorem, one can have
\begin{equation*}
Q_{n}(\theta )\rightarrow _{p}Q_{0}(\theta )\equiv E[m(\mathbf{w}_{i},\theta
)].
\end{equation*}

\item With compactness, how can we show that the convergence is uniform?
\end{itemize}

\end{frame}

\begin{frame}%

\frametitle{Uniform LLN}

\begin{theorem}
Assume that $\left\{ \mathbf{w}_{i}\right\} _{i=1}^{n}$ is ergodic
stationary. Furthermore,

(1) $\Theta $ is compact

(2) $m(\mathbf{w}_{i},\theta )$ is continuous in $\theta ,\forall \mathbf{w}%
_{i}$

(3) $m(\mathbf{w}_{i},\theta )$ is a measurable function of the data, $%
\forall \theta \in \Theta .$

(4) $\exists d(\mathbf{w}_{i})$ such that $\left\vert m(\mathbf{w}%
_{i},\theta )\right\vert \leq d(\mathbf{w}_{i}),$ $\forall \theta \in \Theta
$ and $E[d(\mathbf{w}_{i})]<\infty $

then

(a) convergence is uniform;

(b) $E[m(\mathbf{w}_{i},\theta )]$ is continuous in $\theta .$
\end{theorem}

Use $d(\mathbf{w}_{i})=\sup_{\theta }\left\vert m(\mathbf{w}_{i},\theta
)\right\vert $ and the condition to be verified is $E[\sup_{\theta
}\left\vert m(\mathbf{w}_{i},\theta )\right\vert ]<\infty .$

\end{frame}%
\begin{frame}%

\frametitle{Identification Condition}

\begin{itemize}
\item How about identification condition?

\item For the nonlinear least squares,
\begin{equation*}
m(\mathbf{w}_{i},\theta )=-\left( y_{i}-\varphi (\mathbf{x}_{i},\theta
)\right) ^{2}
\end{equation*}%
and%
\begin{equation*}
E[y_{i}|\mathbf{x}_{i}]=\varphi (\mathbf{x}_{i},\theta ).
\end{equation*}

\item The identification condition holds if
\begin{equation*}
\varphi (\mathbf{x}_{i},\theta _{0})\neq \varphi (\mathbf{x}_{i},\theta
),\forall \theta \neq \theta _{0}.
\end{equation*}
\end{itemize}
\end{frame}

\begin{frame}%

\frametitle{Identification Condition for ML}

\begin{itemize}
\item The Kullback-Leibler information inequality states that
\begin{eqnarray*}
E\left[ \ln \left( \frac{f(\mathbf{w}_{i},\theta )}{f(\mathbf{w}_{i},\theta
_{0})}\right) \right] &\leq &\ln E\left[ \frac{f(\mathbf{w}_{i},\theta )}{f(%
\mathbf{w}_{i},\theta _{0})}\right] \\
&=&0.
\end{eqnarray*}%
\begin{equation*}
\Longleftrightarrow
\end{equation*}%
\begin{equation*}
E[\ln \left( f(\mathbf{w}_{i},\theta )\right) ]\leq E[\ln \left( f(\mathbf{w}%
_{i},\theta _{0})\right) ]
\end{equation*}%
with equality if $\theta =\theta _{0}$ and with strict inequality if $\theta
\neq \theta _{0}$

\item So $E[m(\mathbf{w}_{i},\theta _{0})]$ is uniquely maximized at $\theta
_{0}$ if $f(\mathbf{w}_{i},\theta )\neq f(\mathbf{w}_{i},\theta _{0})$ $%
\forall \theta \neq \theta _{0}.$
\end{itemize}

\end{frame}%
\begin{frame}%

\frametitle{Consistency for ML with compactness}

\begin{theorem}
Suppose that $\left\{ y_{i},\mathbf{x}_{i}\right\} $ is ergodic stationary
with density $f(y_{i}|\mathbf{x}_{i},\theta _{0})$ and $\hat{\theta}$ be the
conditional MLE. The model is assumed to be correctly specified so that $%
\theta _{0}\in \Theta $ and

(1) $\Theta $ is a compact subset $R^{p}.$

(2) $f(y_{i}|\mathbf{x}_{i},\theta )$ is continuous in $\theta ,\forall (%
\mathbf{w}_{i})_{i=1}^{n}$

(3) $f(y_{i}|\mathbf{x}_{i},\theta )$ is a measurable function of the data, $%
\forall \theta \in \Theta .$

(4) (identification) $\Pr \left( f(y_{i}|\mathbf{x}_{i},\theta _{0})\neq
f(y_{i}|\mathbf{x}_{i},\theta )\right) >0,$ $\forall \theta \neq \theta _{0}.
$

(5) (dominance condition) $E_{y_{i},\mathbf{x}_{i}}[\sup_{\theta \in \Theta
}\ln \left\vert \left( f(y_{i}|\mathbf{x}_{i},\theta )\right) \right\vert
]<\infty $

then%
\begin{equation*}
\hat{\theta}\rightarrow _{p}\theta _{0}.
\end{equation*}
\end{theorem}

\end{frame}%

\begin{frame}%

\frametitle{Consistency for ML w/o compactness}

\begin{theorem}
Suppose that $\left\{ y_{i},\mathbf{x}_{i}\right\} $ is ergodic stationary
with density $f(y_{i}|\mathbf{x}_{i},\theta _{0})$ and $\hat{\theta}$ be the
conditional MLE. The model is assumed to be correctly specified so that $%
\theta _{0}\in \Theta $ and

(1) $\theta _{0}\in int\left( \Theta \right) $

(2) $f(y_{i}|\mathbf{x}_{i},\theta )$ is concave in $\theta ,\forall (%
\mathbf{w}_{i})_{i=1}^{n}$

(3) $f(y_{i}|\mathbf{x}_{i},\theta )$ is a measurable function of the data, $%
\forall \theta \in \Theta .$

(4) (identification) $\Pr \left( f(y_{i}|\mathbf{x}_{i},\theta _{0})\neq
f(y_{i}|\mathbf{x}_{i},\theta )\right) >0,$ $\forall \theta \neq \theta _{0}.
$

(5) $E_{y_{i},\mathbf{x}_{i}}[\ln \left\vert \left( f(y_{i}|\mathbf{x}%
_{i},\theta )\right) \right\vert ]<\infty ,$ $\forall \theta $

then%
\begin{equation*}
\hat{\theta}\rightarrow _{p}\theta _{0}.
\end{equation*}
\end{theorem}

\end{frame}%

\begin{frame}%

\frametitle{Asymptotic Normality}

\begin{itemize}
\item Assume that $m(\mathbf{w}_{i},\theta )$ is twice differentiable.

\item Define
\begin{equation*}
\mathbf{s(w}_{i},\theta )_{p\times 1}=\frac{\partial m(\mathbf{w}_{i},\theta
)}{\partial \theta }
\end{equation*}
\end{itemize}

\begin{equation*}
\mathbf{H(w}_{i},\theta )_{p\times p}=\frac{\partial ^{2}m(\mathbf{w}%
_{i},\theta )}{\partial \theta \partial \theta ^{\prime }}
\end{equation*}

\end{frame}%

\begin{frame}%

\frametitle{Asymptotic Normality}

\begin{itemize}
\item Under the assumption that $\hat{\theta}$, $\hat{\theta}$ will be the
solution of FOC:%
\begin{equation*}
0=\frac{\partial Q_{n}(\hat{\theta})}{\partial \theta }=\frac{1}{n}%
\sum_{n=1}^{n}\mathbf{s(w}_{i},\theta ).
\end{equation*} \pause

\item By applying the mean value theorem ($h(x)=h(x_{0})+\frac{\partial h(%
\bar{x})}{\partial x}(x-x_{0}),$ $\bar{x}\in \lbrack x,x_{0}]$),
\begin{eqnarray*}
\frac{\partial Q_{n}(\hat{\theta})}{\partial \theta } &=&\frac{\partial
Q_{n}(\theta _{0})}{\partial \theta }+\frac{\partial ^{2}Q_{n}(\bar{\theta}))%
}{\partial \theta \partial \theta ^{\prime }}(\hat{\theta}-\theta ) \\ \pause
&=&\frac{1}{n}\sum_{i=1}^{n}\mathbf{s(w}_{i},\theta _{0})+\sum_{i=1}^{n}%
\mathbf{H(w}_{i},\bar{\theta})(\hat{\theta}-\theta _{0}) \\
&=&0. \pause
\end{eqnarray*}
\item Notice that $E[\bm{s}(\bm{w}_i,\theta_0)]=0$ so by CLT:
  \[\frac{1}{n}\sum_{i=1}^{n}\mathbf{s(w}_{i},\theta _{0}) \rightarrow_d \mc{N}(0,\Sigma)\]
  with $\Sigma = E[\bm{s}(\bm{w}_i,\theta_0)\bm{s}(\bm{w}_i,\theta_0)']$.
\end{itemize}

\end{frame}%
\begin{frame}%

\frametitle{Asymptotic Normality}

\begin{itemize}
\item Assuming that $\frac{1}{n}\sum_{i=1}^{n}\mathbf{H(w}_{i},\bar{\theta})$
is nonsingular,
\begin{equation*}
\sqrt{n}(\hat{\theta}-\theta _{0})=-\left[ \frac{1}{n}\sum_{i=1}^{n}\mathbf{H(w}_{i},\bar{\theta})\right] ^{-1}\frac{1}{\sqrt{n}}\sum_{i=1}^{n}\mathbf{s(w}_{i},\theta _{0}).
\end{equation*} \pause

\item If
\begin{equation*}
\frac{1}{n}\sum_{i=1}^{n}\mathbf{H(w}_{i},\bar{\theta})\rightarrow _{p}E[\mathbf{H(w}_{i},\theta _{0})]
\end{equation*}%
and
\begin{equation*}
\frac{1}{\sqrt{n}}\sum_{i=1}^{n}\mathbf{s(w}_{i},\theta _{0})\rightarrow
_{d}N(\mathbf{0,}\Sigma )
\end{equation*}% \pause
then by Slutzky theorem%
\begin{equation*}
\sqrt{n}(\hat{\theta}-\theta _{0})\rightarrow _{d}N(\mathbf{0,}E[\mathbf{H(w}%
_{i},\theta _{0})]^{-1}\Sigma E[\mathbf{H(w}_{i},\theta _{0})]^{-1})
\end{equation*}
\end{itemize}

\end{frame}%
\begin{frame}%

\frametitle{Asymptotic Normality for M-estimator}

\begin{theorem}
Assume that conditions for consistency hold. Also,

\begin{enumerate}
\item  $\theta _{0}\in int\left( \Theta \right) $ \pause

\item  $m(\mathbf{w}_{i},\theta )\in C^{2}$ in $\theta ,\forall \mathbf{w}_{i}$ \pause

\item  $\sqrt{n}\frac{1}{n}\sum_{i=1}^{n}\mathbf{s(w}_{i},\theta )\rightarrow
_{d}N(\mathbf{0},\Sigma )$ where $\Sigma $ is positive def. \pause

\item  (local dominance for the Hessian)%
\begin{equation*}
E\left( \sup_{\theta \in N(\theta _{0})}\left\Vert \mathbf{H(w}_{i},\theta
)\right\Vert \right) <\infty .
\end{equation*} \pause

\item  $E\left( \mathbf{H(w}_{i},\theta _{0})\right) $ is nonsingular. \pause
\end{enumerate} \pause

then%
\begin{equation*}
\sqrt{n}\left( \theta -\theta _{0}\right) \rightarrow _{d}N(\mathbf{0}%
,E\left( \mathbf{H(w}_{i},\theta _{0})\right) ^{-1}\Sigma E\left( \mathbf{H(w%
}_{i},\theta _{0})\right) ^{-1})
\end{equation*}
\end{theorem}

\end{frame}%

\begin{frame}\frametitle{Fisher Information Matrix}
  We can show that when $m(\textbf{w}_i,\theta)=\log(f(y_i|\textbf{x}_i,\theta)$:
  \[\Sigma = E[\mathbf{s}(\mathbf{w}_i,\theta_0)\mathbf{s}(\mathbf{w}_i,\theta_0)^\prime] = -E[\mathbf{H}(\mathbf{w}_i,\theta_0)]\]
  So asympotic variance is:
\begin{equation*}
Avar(\hat{\theta})=-E\left( \mathbf{H(w}_{i},\theta _{0})\right) ^{-1}=E[\mathbf{s(w}_{i},\theta_0)\mathbf{s(w}_{i},\theta )^{\prime }]^{-1}=\mathcal{I}(\theta)^{-1}.
\end{equation*}
Sometimes call $\mathcal{I}(\theta)$ the \emph{Fisher} information matrix. It's the covariance of the score, $\mathbf{s}_i$.
\end{frame}

\begin{frame}\frametitle{Efficiency}
\begin{itemize}
\item ML can be thought of as efficient among a number of important classes of estimators. \pause
\item Efficient among GMM estimators. \pause
\item $\mathcal{I}(\theta)^{-1}$ is the \emph{Cramer-Rao lower bound} for all unbiased estimators of $\theta_0$ \pause
\item One reason MLE is efficient: it makes use of every available piece of information. \pause
\item This means it takes every feature of the model \emph{very} seriously. \pause
\item Statistical information $\neq$ theoretical or conceptual importance. \pause
\item Leans heavily on specification of parametric distributions.
\end{itemize}

\end{frame}

\begin{frame}\frametitle{Efficiency relative to GMM}
  Recall $\hat{\theta}_{GMM}=\arg\min g_n(\theta)'Wg_n(\theta)$ where
  \[g_n = \frac{1}{n}\sum g(z_i,\theta)\]
  You know that
  \[V[\hat{\theta}_{GMM}] = E[m_\theta]^{-1}E[mm']E[m_\theta]^{-1} \]
  where $m_\theta=E[\nabla_\theta g]^\prime W\nabla_\theta g(z,\theta_0)'$ and $m=E[\nabla_\theta g]^\prime W g(z,\theta_0)$.
\end{frame}

\begin{frame}\frametitle{Efficiency relative to GMM}
  Now use the identity:
    \begin{align*}
      0 &= \int g(z,\theta)f(z|\theta)dz\ (\text{identity for all $\theta$}) \\ \pause
      \Rightarrow      0 &= \int \nabla_\theta g(z,\theta)f(z|\theta)dz + \int g(z,\theta) \nabla_\theta\log(f(z|\theta))f(z|\theta) dz \\ \pause
      0 &=  E[\nabla_\theta g(z,\theta)] - E[g(z,\theta)\mathbf{s}(z,\theta)^\prime] \\
   \Leftrightarrow   0 &= E[m_\theta] - E[m\mathbf{s}(z,\theta)^\prime]
    \end{align*} \pause
    This gives:
    {\footnotesize
\begin{align*}
  V[\hat{\theta}_{GMM}]-V[\hat{\theta}_{MLE}] &= E[m_\theta]^{-1}E[mm^\prime]E[m_\theta]^{-1} - E[ss^\prime]^{-1} \\ \pause
                                              &= E[ms^\prime]^{-1}E[mm^\prime]E[ms^\prime]^{-1} - E[ss^\prime]^{-1} \\ \pause
                                              &= E[ms']^{-1}(E[mm^\prime] - E[ms^\prime]E[ss^\prime]^{-1}E[ms^\prime])E[ms^\prime]^{-1} \\ \pause
                                                &= E[ms']^{-1}E[UU']E[ms']^{-1} \geq 0
\end{align*}}
where $U=m-E[ms']E[ss']^{-1}s$.
\end{frame}

%% HERE: put in application to a simple search model!!
\begin{frame}\frametitle{Two Applications}
\begin{enumerate}
\item Stationary search model.
\begin{itemize}
\item Parameters $b$, $F_w$, $\delta$, $\lambda$.
\item Single cross-section on wages and unemployment durations.
\item Review case on board. We'll return to this example in future sections.
\item With and without observable heterogeneity, $\mathbf{x}_i$.
\end{itemize} \pause
\item Life-cycle savings model.
  \begin{itemize}
  \item Preferences: $(\beta,\sigma)$
  \item Income process: \[y_{t} = \mu_{t} + \varepsilon_{t}+v_{t},\ \varepsilon_{t+1}=\rho\varepsilon_{t}+\xi_{t+1},\ v_t\sim\mc{N}(0,\sigma^2_{v}),\ \xi_t\sim\mc{N}(0,\sigma^2_{\xi})\]
  \item Initial conditions: $\varepsilon_0\sim\mc{N}(0,\sigma^2_{\xi}/(1-\rho^2))$, $a_0=0$.
  \item With short panels, full likelihood here is tricky!
  \end{itemize}
  % \item Generic model of choice.
% \begin{itemize}
% \item ${\Psi}(\pi,\theta)=0$ where $\pi$ are choice probabilities, $\dim(\pi)=D$.
% \item $\Psi$ are optimality conditions (e.g. dynamic program).
% \item Data on choices, $d_i\in\{0,...,D-1\}$.
% \item Model produces mapping $\pi = \pi(\theta)$.
% \end{itemize}
\end{enumerate}
Let's write the likelihood for both cases.
\end{frame}


\begin{frame}\frametitle{Minimum Distance Estimator}
\begin{definition}[Minimum Distance Estimator]
  Suppose:
  \begin{itemize}
  \item $\hat{\pi}_n$ is a statistic from the data, and
  \item $\psi(\pi,\theta)$ is a vector of model restrictions.
  \item Idea: $\psi(\pi_0,\theta_0)=0$ and $\pi_n\rightarrow_p\pi_0$. \pause
  \end{itemize}
  $\hat{\theta}_n$ is a \textbf{minimum distance estimator} if:
  \[\hat{\theta}_n=\arg\min_{\theta} \psi(\hat{\pi}_n,\theta)'\mathbf{W}_n\psi(\hat{\pi}_n,\theta)\]
\end{definition} \pause
Note:
\begin{itemize}
\item There is more general definition that nests GMM as special case. \pause
\item For consistency, apply theorem of choice for extremum estimators.
\item   Let $\nabla_\theta \psi_0 = \frac{\partial \psi(\theta_0,\pi_0)'}{\partial \theta}$ and $\nabla_\theta \psi_n =  \frac{\partial \psi(\hat{\theta}_n,\hat{\pi}_n)'}{\partial \theta}$
\end{itemize}
\end{frame}


\begin{frame}\frametitle{Asymptotics for MD}
  \begin{theorem}[Asymptotics for MD]
    Suppose:
    \begin{enumerate}
    \item $\psi(\pi_0,\theta_0) = 0$, $\psi(\pi_0,\theta)\neq 0\ \forall\theta\neq\theta_0$. \pause
    \item $\sqrt{n}(\hat{\pi}_n-\pi_0)\rightarrow_d\mathcal{N}(0,\bm{\Omega})$ \pause
    \item $\mathbf{W}_n\rightarrow_p \mathbf{W}$, each symmetric, nonsingular. \pause
    \item $\psi$ differentiable, $\text{rank}(\nabla_\theta \psi)= p$. \pause
    \end{enumerate}
    then:
    $\sqrt{n}(\hat{\theta}_n-\theta_0)\rightarrow_d\mathcal{N}(0,\Sigma)$ with
    \[\Sigma = (\nabla_\theta\psi_0 \mathbf{W} \nabla_\theta\psi_0')^{-1}\nabla_\theta\psi_0 \mathbf{W}\nabla_\pi\psi_0'\bm{\Omega}\nabla_\pi\psi_0 \mathbf{W} \nabla_\theta \psi_0'(\nabla_\theta\psi_0 \mathbf{W} \nabla_\theta\psi_0')^{-1}\] \pause
  \end{theorem}
Efficient weighting: $\mathbf{W}=(\nabla_\pi\psi_0'\bm{\Omega}\nabla_\pi\psi_0)^{-1}$, gives:
\[\Sigma^* = (\nabla_\theta\psi_0 (\nabla_\pi\psi_0'\bm{\Omega}\nabla_\pi\psi_0)^{-1} \nabla_\theta\psi_0')^{-1} \]
\end{frame}

\begin{frame}\frametitle{Asymptotics for MD}
  Delta Method:
  \begin{align*}
    \psi(\hat{\pi}_n,\hat{\theta}_n) &\approx \psi(\pi_0,\theta_0) + \nabla_\pi\psi'_0(\hat{\pi}_n-\pi_0) + \nabla_{\theta}\psi'_0(\hat{\theta}_n-\theta_0) \\\pause
    \underbrace{\nabla_\theta\psi_n\mathbf{W}_n\psi(\hat{\pi}_n,\hat{\theta}_n)}_{=0} &\approx \nabla_\theta\psi_n\mathbf{W}_n(\nabla_\pi\psi_0'(\hat{\pi}_n-\pi_0) + \nabla_{\theta}\psi_0(\hat{\theta}_n-\theta_0))  \\ \pause
   \hat{\theta}_n-\theta_0 &\approx (\nabla_\theta\psi_n \mathbf{W}_n\nabla_\theta\psi_0')^{-1}\nabla_\theta\psi_n \mathbf{W}_n\nabla_\pi\psi_0'(\hat{\pi}_n-\pi)
  \end{align*}
  \begin{itemize}
  \item Applying Slutsky gives the result.
  \item Could have also used MVT. In this case, $\approx$ used because of second order terms that disappear.
  \item Note the special case: $\pi - \psi(\theta)=0$. A bit simpler.
  \item Often, $\psi(\pi,\theta)$ defines such an \emph{implicit function}.
  \end{itemize}
\end{frame}

% get CPS data, and have them estimate two ways!
% this would be a nice application

% \begin{frame}\frametitle{Applications of Minimum Distance}
% \begin{enumerate}
% \item Search Model
% \begin{itemize}
% \item Previous case: solve reservation wage for each choice of $\theta$. \pause
% \item But we \emph{could} estimate reservation wages directly (turns out that $w^*=\min_i w_i$ ) \pause
% \item No weighting necessary here. \pause
% \end{itemize}
% \item Choice Model
% \begin{itemize}
% \item Recall that $\Psi$ is vector of optimality conditions. \pause
% \item Suppose that $\Psi$ is easy to compute (i.e. check) but not easy to solve. \pause
% \item Solution: estimate $\pi$ directly and do minimum distance!! \pause
% \item $\hat{\pi}_{n,d} = \frac{1}{n}\sum_{i=1}^n\mathbf{1}\{d_i=d\}$ (frequency estimator) \pause
% \item Just as efficient as MLE when optimally weighted. \pause
% \end{itemize}
% \end{enumerate}
% \end{frame}

\begin{frame}\frametitle{Efficiency of Optimal Minimum Distance}
  Let $\dim(\psi)=\dim(\pi)$, IFT says:
  \[\nabla_\theta \pi = -\nabla_\theta\psi (\nabla_\pi\psi')^{-1} \]
  So:
  \begin{align*}
    \Sigma^* &= (\nabla_\theta\psi_0(\nabla_\pi\psi_0' E[s_\pi s_\pi']^{-1} \nabla_\pi \psi_0)^{-1}\nabla_\theta \psi_0')^{-1} \\
            &= (\nabla_\theta \psi_0\nabla_\pi\psi_0'^{-1}E[s_\pi s_\pi'] \nabla_\pi \psi_0^{-1}\nabla_\theta \psi_0')^{-1} \\
            &= (\nabla_\theta \pi E[s_\pi s_\pi']\nabla_\theta \pi')^{-1} \\
            &= E[s_\theta s_\theta']^{-1} = \mathcal{I}_\theta
  \end{align*}
  If $\dim(\psi)>\dim(\pi)$, use \emph{pseudo-inverse} of $\nabla_\pi\psi_0$.
\end{frame}

\begin{frame}\frametitle{Two Step Estimators}
  Setup, two parameter sets: $\gamma$, $\beta$. $\gamma$ estimated by:
  \[\frac{1}{n}\sum g_1(\mathbf{w}_i,\h{\gamma}) = \mathbf{0}\] \pause
  And then $\beta$ estimated by:
  \[\frac{1}{n}\sum g_2(\mathbf{w}_i,\h{\gamma},\h{\beta}) = \mathbf{0}\] \pause
  Note:
  \begin{itemize}
  \item This is GMM (MLE nested).
  \item $g_1$ and $g_2$ incorporate weighting, so apply wlog.
  \item $\dim(g_1)=\dim(\gamma)$, $\dim(g_2)=\dim(\beta)$.
  \end{itemize}
  \pause Questions:
  \begin{enumerate}
  \item Is $\h{\beta}$ consistent?
  \item How does variance of $\h{\gamma}$ contribute to variance of $\h{\beta}$?
  \end{enumerate}

\end{frame}

\begin{frame}\frametitle{Two Step Estimators: Asymptotics}
  Define $\theta = [\gamma',\ \beta']'$, and
  \begin{align*}
    g(\mathbf{w}_i,\theta) &= \left[\barr{c}g_1(\mathbf{w}_i,\gamma) \\ g_2(\mathbf{w}_i,\gamma,\beta) \earr\right]  \\
    \bm{\Omega} &= E[g(\mathbf{w}_i,\theta_0)g(\mathbf{w}_i,\theta_0)'] \\
    E[\nabla_\theta g'] = \bm{\Gamma} &= \left[\barr{cc} \Gamma_{1\gamma} & \mathbf{0} \\\Gamma_{2\gamma} & \Gamma_{2\beta}\earr\right]
  \end{align*}  \pause
  Under regularity conditions:
  \[\sqrt{n}(\hat{\theta}-\theta_0) \rightarrow_d \mc{N}(0,\bm{\Gamma}'^{-1}\bm{\Omega}\bm{\Gamma}^{-1})\] \pause
    which means:
    \[\sqrt{n}(\h{\beta}-\beta_0) \rightarrow_d \mc{N}(0,(\bm{\Gamma}'^{-1}\bm{\Omega}\bm{\Gamma}^{-1})_{22})\]
\end{frame}

\begin{frame}\frametitle{Two Step Estimators: Asymptotics}
  Partitioned inverse formula gives:
  \[\bm{\Gamma}^{-1} = \left[\barr{cc} \Gamma_{1\gamma}^{-1} & \mathbf{0} \\ -\Gamma^{-1}_{2\beta}\Gamma_{2\gamma}\Gamma^{-1}_{1\gamma} & \Gamma^{-1}_{2\beta}\earr\right]\] \pause
  ...Algebra... \pause
  \begin{align*}
    \BB{V}[\sqrt{n}\h{\beta}] =& \Gamma_{2\beta}^{-1}{\color{blue}[\Omega_{22}  - \Gamma_{2\gamma}\Gamma_{1\gamma}^{-1}\Omega_{12} - \Omega_{12}'\Gamma_{1\gamma}'\Gamma_{2\gamma}'^{-1}} \\ &{\color{blue}+ \Gamma_{2\gamma}\Gamma_{1\gamma}^{-1}\Omega_{11}\Gamma_{1\gamma}'^{-1}\Gamma_{2\gamma}']}\Gamma_{2\beta}'^{-1} \\
    =& \Gamma_{2\beta}^{-1}{\color{blue}\BB{V}[g_2(\mathbf{w}_i,\theta_0) - \Gamma_{2\gamma}\Gamma^{-1}_{1\gamma}g_1(\mathbf{w}_i,\gamma_0)]}\Gamma_{2\beta}'^{-1}
  \end{align*} \pause
So, no contribution if $\Gamma_{2\gamma}=E[\nabla_\gamma g_2(\mathbf{w}_i,\theta_0)']=\mathbf{0}$
\end{frame}

\begin{frame}\frametitle{Application: Two-stage IV}
\begin{eqnarray}
  y_i = \alpha_0 + D_i\beta_0 + u_i \\ \pause
  P[D_i=1|\mathbf{x}_i] = \Phi(\mathbf{x}_i'\gamma_0) \\ \pause
  E[u_i|\mathbf{x}_i]=0 \pause
\end{eqnarray}
So $X_i$ are instruments for $D_i$. \pause Estimate by:
\begin{enumerate}
\item Get $\h{\gamma}$ by MLE.
\item Do GMM using:
  \[E\left[\left(\barr{c}1 \\ \Phi(\mathbf{x}_i'\gamma_0)\earr\right)u_i\right] = \mathbf{0}\]
\end{enumerate}
\end{frame}

\begin{frame}\frametitle{Application: Two-stage IV}
  Here:
  \begin{eqnarray}
    g_1(\mathbf{w}_i,\gamma) = \frac{(D_i-\Phi(\mathbf{x}_i'\gamma))\phi(\mathbf{x}_i'\gamma)}{\Phi(\mathbf{x}'_i\gamma)(1-\Phi(\mathbf{x}'_i\gamma))}\mathbf{x}_i \nonumber \\
    g_2(\mathbf{w}_i,\theta) = \left(\barr{c}1 \\ \Phi(\mathbf{x}_i'\gamma)\earr\right)(y_i-\alpha-D_i\beta) \nonumber \pause
  \end{eqnarray}
  Since $E[\nabla_\gamma g_2(\mathbf{w}_i,\theta_0)']=0$, we can ignore contribution of $\h{\gamma}$ to variance. (Exercise on board).
\end{frame}

\begin{frame}\frametitle{Auxiliary Information}
  Assume:
  \begin{eqnarray}
    E[g_1(\mathbf{w}_i)] = 0 \nonumber\\
    E[g_2(\mathbf{w}_i,\theta)] = 0 \nonumber
  \end{eqnarray}
  Just identified, wlog.
  Cases:
  \begin{enumerate}
  \item Estimate $\theta$ using $g_2$ and ignoring $g_1$.
  \item Estimate $\theta$ using $g=[g_1',\ g_2']'$, with optimal weighting.
  \end{enumerate}
  Notation:
  \begin{eqnarray}
    \Gamma_0 = E[\nabla_\theta g_2(\mathbf{w}_i,\theta_0)'],\qquad \Gamma_* = \left[\barr{c}\mathbf{0} \\ \Gamma_0\earr\right]\nonumber \\
    E[g(\mathbf{w}_i,\theta_0)g(\mathbf{w}_i,\theta_0)'] = \Omega_* = \left[\barr{cc}\Sigma & C' \\ C & \Omega\earr\right] \nonumber
  \end{eqnarray}
\end{frame}

\begin{frame}\frametitle{Efficiency under Auxiliary Information}
Asymptotic variances are:
\begin{eqnarray}
  Avar_1 = (\Gamma_0\Omega^{-1}\Gamma_0')^{-1} \nonumber \\
  Avar_2 = (\Gamma_*\Omega_*^{-1}\Gamma_*')^{-1} \nonumber
\end{eqnarray}
Paritioned inverse formula + algebra:
\[\Gamma_*\Omega_*^{-1}\Gamma'_* = \Gamma_0(\Omega \underbrace{- C\Sigma^{-1}C'}_{\color{blue} \text{Gain in precision from }g_1 })^{-1}\Gamma'_0 \]
Use this to show that $Avar_1 - Avar_2 \geq 0$. Notes:
\begin{itemize}
\item Punchline: auxiliary information on data (even without parameter restrictions) can lead to increases in efficiency.
\item Example: estimating a regression model on micro data, when macro data tells you \emph{almost} exactly the value of some marginal distributions.
\item Logic extends to adding moment restrictions.
\item Lancaster \& Imbens (1994)
\end{itemize}
\end{frame}
\end{document}

%%% Local Variables:
%%% mode: latex
%%% TeX-master: t
%%% End:
